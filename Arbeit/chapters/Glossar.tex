\makeglossaries 

\newglossaryentry{framework}
{
	name={Framework},  
	plural={Frameworks},
	description={Programmgerüst ohne anwendungsspezifische Sotfwareteile. \citep[S. 847]{gumm2011einfuhrung}}
}  
\newglossaryentry{unittest}
{
	name={Unittest},  
	plural={Unittests},
	description={Funktionstest des Entwicklers zum Testen einer klar eingegrenzten Funktionalität. \citep[S. 3]{hunt2004unit}}
}  
\newglossaryentry{dateideskriptor}
{
	name={Dateideskriptor},  
	plural={Dateideskriptoren},
	description={Integerzahl, die der Prozess beim erfolgreichen Öffnen einer Datei für weitere Referenzierungen auf diese Datei zurückerhält. \citep[S. 25]{beck1994linux}}
}  
\newglossaryentry{handle}
{
	name={Handle},  
	plural={Handles},
	description={Referenz auf eine bestimmte Ressource.}
} 
\newglossaryentry{middleware}
{
	name={Middleware},  
	description={Universell einsetzbarer Softwareteil. \citep[S. 487]{sommerville2011software}}
} 
\newglossaryentry{frame}
{
	name={Frame},
	plural={Frames},  
	description={Einzelbild eines Videos.}
} 