\chapter{Fazit}
% Transaktionsids
% Bild Anton
%Die Abhängigkeit von einem händisch eingetragenen Offset ist so immer noch vorhanden!!!


% weitere vorgehensweise, weitere module, speicherbereiche, arbeitsaufwand: ca halbes jahr

In diesem Abschnitt soll auf die weitere Vorgehensweise ...

\section{Zusammenfassung der Ergebnisse}


\section{Ausblick}













%In diesem Abschnitt soll auf die weitere Verwendung des \ac{geo} eingegangen werden und ein Ausblick auf die Zukunft gegeben werden.

%\section{Zum \ac{geo}}
%Zu Beginn von Kapitel \ref{sec:haupt} wurde schon erwähnt, dass es mehr Module als die vier in der Arbeit betrachteten Module gibt. Aktuell gibt es um die zehn verschiedenen Module, allerdings werden, außer zu Entwicklungszwecken, keine neuen Module dazukommen. 
%In der Kamerasoftware existieren noch viele Stellen an der die betrachtete Abstraktionsebene gar nicht oder nicht vollständig zum Einsatz kommt. Natürlich soll das \ac{geo} an allen möglichen Stellen verwendet werden. Aus diesem Grund müssen die unterschiedlichen Module entweder einzeln oder übergreifend um neue Parameter erweitert werden. Auch die modulspezifischen Abhängigkeiten müssen weiter angepasst und gepflegt werden. 

%Insbesondere müssen hardwareseitige Änderungen berücksichtigt werden und entsprechend softwareseitig die Module nachgezogen werden.
%\section{Über die betrachtete Abstraktionsebene hinaus}
%Zur weiteren Verbesserung der Software soll es in Zukunft noch eine weitere Abstraktionsebene geben. Diese sogenannte \ac{fra} soll dann allerdings nicht wie das \ac{geo} die Bildkette abbilden, sondern die Hardwarefunktionen aus der Hauptcode elimentieren und damit für noch mehr Wartbarkeit sorgen. 

%Durch das Zusammenspiel der beiden Abstraktionsebenen soll ohne tiefere Kenntnisse von \ac{FPGA} und Bildkette das Pflegen und Erweitern der Kamerasoftware durch unterschiedliche Entwicklern ermöglicht werden.