\chapter{Fazit und Ausblick}
% Transaktionsids
% Bild Anton
%Die Abhängigkeit von einem händisch eingetragenen Offset ist so immer noch vorhanden!!!
% weitere vorgehensweise, weitere module, speicherbereiche, arbeitsaufwand: ca halbes jahr


\section{Zusammenfassung der Ergebnisse}
%letzes framework zum umstrukturieren der software, 
% abstraktion zur hardware, kein direkter zugriff mehr
% testen möglich
% über verschiedene ebenen

Ein wichtiger Punkt für das \ac{fra} war die Abstraktion der Hardware. Durch die direkten Zugriffe von der Kamerasoftware auf den \ac{fpga} ist die Fehlersuche recht zeitaufwendig gewesen. Aufgrund des neuen \glspl{framework} wird nicht mehr über direkte Adressen auf die Hardware zugegriffen. Dadurch ist es einfacher nachzuvollziehen, aus welchem Prozess die Funktions\-aufrufe kommen und kann so im Fehlerfall auch einfach protokolliert werden.


Des Weiteren wird durch die Einteilung in drei verschiedene Ebenen im Userspace eine einfache Möglichkeit gegeben, Tests zu implementieren. Durch das Test-Backend und die \ac{fra}-Middleware können die modulspezifischen Funktionen identisch wie in der Kamerasoftware aufgerufen werden. Somit kann überprüft werden, ob die \ac{fra}-Middleware und -Bibliothek richtig funktionieren. So können durch die \glspl{unittest} bei jeder Neuerung und Änderung in der Bibliothek oder Middleware frühzeitig Fehler gefunden und behoben werden.


Durch das \ac{fra} ist der letzte Schritt in einer Umstrukturierung der Software gemacht worden. Die letzten Jahre wurde die Software bereits funktional durch die Implementierung des \ac{geo} abstrahiert. Aufgrund des Zusammenspiels der beiden Abstraktionsebenen ist es nun möglich, ohne tiefere Kenntnisse von \ac{fpga} und Bildkette die Kamerasoftware durch unterschiedliche Entwickler zu pflegen und zu erweitern.


\section{Ausblick}
%weitere vorgehenesweise: restliche module umziehen, speicherbereiche im fpga von modulen umziehen: ca halbes jahr
%tests weiter ausbauen

%abhängigkeit von händischen offset noch vorhanden, muss noch raus
% transids mit funktionalität belegen

Damit das \ac{fra} vollständig in der Software eingebunden ist, müssen noch weitere Schritte durchgeführt werden. Die restlichen Module der Bildkette samt ihrer spezifischen Funktionen müssen in die \ac{fra}-Bibliothek migriert werden. Der Arbeitsaufwand für die komplette Migration der Software auf das \gls{framework} wird auf ungefähr ein halbes Jahr geschätzt.
% und zum Anderen haben ein paar Module in \ac{fpga} Speicherbereiche um die Bilder zwischen zu speichern, auch diese müssen für die komplette Umstellung im \ac{fra} abgebildet werden.

Auch die Abhängigkeit von händisch eingetragenen Adressen ist noch immer vorhanden (siehe Kapitel~\ref{sec:konzept}). Diese soll in Zukunft durch eine Generierung aus der \ac{fpga}-Bildkette ersetzt werden und somit eine komplette Unabhängigkeit von den \ac{fpga} Adressen in der Software schaffen. Dies ist vor allem in Anbetracht der Wartbarkeit der Software ein wichtiges Thema.

Wie in Kapitel~\ref{sec:middleware} erwähnt, wird in der aktuellen Implementierung des \ac{fra} die Transaktionsidentifikation durch alle Funktionsaufrufe durchgereicht. Bei bestimmten Pfaden der Bildkette hängen die Einstellungen der Module stark voneinander ab, aus diesem Grund ist es sinnvoll, die Zugriffe auf die Module zu gruppieren. Dadurch können die \ac{fpga}-Zugriffe direkt hintereinander aufgerufen und so Bildfehler vermieden werden.
Die Erweiterung ist relativ zeitaufwendig und auch mit Änderungen im Kernel verbunden.


Der aktuelle Stand des \ac{fra} ist bereits funktional in der Software integriert und muss noch über die obigen Teile erweitert werden, damit die Umstrukturierung komplett abgeschlossen ist.


















%In diesem Abschnitt soll auf die weitere Verwendung des \ac{geo} eingegangen werden und ein Ausblick auf die Zukunft gegeben werden.

%\section{Zum \ac{geo}}
%Zu Beginn von Kapitel \ref{sec:haupt} wurde schon erwähnt, dass es mehr Module als die vier in der Arbeit betrachteten Module gibt. Aktuell gibt es um die zehn verschiedenen Module, allerdings werden, außer zu Entwicklungszwecken, keine neuen Module dazukommen. 
%In der Kamerasoftware existieren noch viele Stellen an der die betrachtete Abstraktionsebene gar nicht oder nicht vollständig zum Einsatz kommt. Natürlich soll das \ac{geo} an allen möglichen Stellen verwendet werden. Aus diesem Grund müssen die unterschiedlichen Module entweder einzeln oder übergreifend um neue Parameter erweitert werden. Auch die modulspezifischen Abhängigkeiten müssen weiter angepasst und gepflegt werden. 

%Insbesondere müssen hardwareseitige Änderungen berücksichtigt werden und entsprechend softwareseitig die Module nachgezogen werden.
%\section{Über die betrachtete Abstraktionsebene hinaus}
%Zur weiteren Verbesserung der Software soll es in Zukunft noch eine weitere Abstraktionsebene geben. Diese sogenannte \ac{fra} soll dann allerdings nicht wie das \ac{geo} die Bildkette abbilden, sondern die Hardwarefunktionen aus der Hauptcode elimentieren und damit für noch mehr Wartbarkeit sorgen. 

%Durch das Zusammenspiel der beiden Abstraktionsebenen soll ohne tiefere Kenntnisse von \ac{FPGA} und Bildkette das Pflegen und Erweitern der Kamerasoftware durch unterschiedliche Entwicklern ermöglicht werden.