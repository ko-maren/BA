\chapter{Fazit und Ausblick}
% Transaktionsids
% Bild Anton
%Die Abhängigkeit von einem händisch eingetragenen Offset ist so immer noch vorhanden!!!
% weitere vorgehensweise, weitere module, speicherbereiche, arbeitsaufwand: ca halbes jahr

In diesem Kapitel soll ein Fazit über das in der Arbeit behandelte \ac{fra} gezogen werden und zusätzlich ein Ausblick auf die weitere Vorgehensweise zur Einbindung in die Software und notwendige Erweiterungen gegeben werden.


\section{Zusammenfassung der Ergebnisse}
%letzes framework zum umstrukturieren der software, 
% abstraktion zur hardware, kein direkter zugriff mehr
% testen möglich
% über verschiedene ebenen

Ein wichtiger Punkt für das \ac{fra} war die Abstraktion der Hardware. Durch die direkten Zugriffe von der Kamerasoftware auf den \ac{fpga} ist die Fehlersuche recht zeitaufwendig gewesen. Durch das neue Framework wird nicht mehr über direkte Adressen auf die Hardware zugegriffen. Dadurch ist es einfacher nachzuvollziehen, aus welchem Prozess die Funktionsaufrufe kommen und kann so im Fehlerfall auch einfach protokolliert werden.


Des weiteren wird durch die Einteilung in drei verschiedene Ebenen im Userspace eine einfache Möglichkeit gegeben um Tests zu implementieren. Durch das Testbackend und die \ac{fra} Middleware können die modulspezifischen Funktionen identisch wie in der Kamerasoftware aufgerufen werden und somit überprüft werden, ob die \ac{fra} Bibliothek richtig funktioniert. So können durch die Unittests bei jeder Neuerung und Änderung in der Bibliothek oder Middleware frühzeitig Fehler gefunden und behoben werden.


Durch das \ac{fra} ist der letzte Schritt in einer Umstrukturierung der Software gemacht worden. Die letzten Jahre wurde die Software bereits funktional durch die Implementierung des \ac{geo} abstrahiert. Durch das Zusammenspiel der beiden Abstraktionsebenen ist es nun möglich ohne tiefere Kenntnisse von \ac{fpga} und Bildkette die Kamerasoftware durch unterschiedliche Entwickler zu Pflegen und zu Erweitern.


\section{Ausblick}
%weitere vorgehenesweise: restliche module umziehen, speicherbereiche im fpga von modulen umziehen: ca halbes jahr
%tests weiter ausbauen

%abhängigkeit von händischen offset noch vorhanden, muss noch raus
% transids mit funktionalität belegen

Damit des \ac{fra} vollständig in der Software eingebunden ist, müssen noch weitere Schritte gemacht werden. Zum Einen müssen die restlichen Module der Bildkette samt ihrer spezifischen Funktionen in die \ac{fra} Bibliothek umgezogen werden und zum Anderen haben ein paar Module in \ac{fpga} Speicherbereiche um die Bilder zwischen zu speichern, auch diese müssen für die komplette Umstellung im \ac{fra} abgebildet werden. Der Arbeitsaufwand für den kompletten Umzug der Software auf das Framework wird auf ungefähr ein halbes Jahr geschätzt.

Auch die Abhängigkeit von händisch eingetragenen Adressen ist noch immer vorhanden (siehe Kapitel~\ref{sec:konzept}). Diese soll in Zukunft durch ein generisches Modell ersetzt werden und somit eine komplette Unabhängigkeit von den \ac{fpga} Adressen in der Software schaffen. Dies ist vor allem in Anbetracht der Wartbarkeit der Software ein wichtiges Thema.

Wie in Kapitel~\ref{sec:middleware} erwähnt, wird in der aktuellen Implementierung des \ac{fra} die Transaktionsidentifikation durch alle Funktionsaufrufe durchgereicht. Bei bestimmten Teilen der Bildkette hängen die Einstellungen der Module stark voneinander ab, aus diesem Grund ist es sinnvoll die Zugriffe auf die Module zu gruppieren. Dadurch können die \ac{fpga} Zugriffe direkt hintereinander aufgerufen werden und so Bildfehler vermieden werden.


















%In diesem Abschnitt soll auf die weitere Verwendung des \ac{geo} eingegangen werden und ein Ausblick auf die Zukunft gegeben werden.

%\section{Zum \ac{geo}}
%Zu Beginn von Kapitel \ref{sec:haupt} wurde schon erwähnt, dass es mehr Module als die vier in der Arbeit betrachteten Module gibt. Aktuell gibt es um die zehn verschiedenen Module, allerdings werden, außer zu Entwicklungszwecken, keine neuen Module dazukommen. 
%In der Kamerasoftware existieren noch viele Stellen an der die betrachtete Abstraktionsebene gar nicht oder nicht vollständig zum Einsatz kommt. Natürlich soll das \ac{geo} an allen möglichen Stellen verwendet werden. Aus diesem Grund müssen die unterschiedlichen Module entweder einzeln oder übergreifend um neue Parameter erweitert werden. Auch die modulspezifischen Abhängigkeiten müssen weiter angepasst und gepflegt werden. 

%Insbesondere müssen hardwareseitige Änderungen berücksichtigt werden und entsprechend softwareseitig die Module nachgezogen werden.
%\section{Über die betrachtete Abstraktionsebene hinaus}
%Zur weiteren Verbesserung der Software soll es in Zukunft noch eine weitere Abstraktionsebene geben. Diese sogenannte \ac{fra} soll dann allerdings nicht wie das \ac{geo} die Bildkette abbilden, sondern die Hardwarefunktionen aus der Hauptcode elimentieren und damit für noch mehr Wartbarkeit sorgen. 

%Durch das Zusammenspiel der beiden Abstraktionsebenen soll ohne tiefere Kenntnisse von \ac{FPGA} und Bildkette das Pflegen und Erweitern der Kamerasoftware durch unterschiedliche Entwicklern ermöglicht werden.