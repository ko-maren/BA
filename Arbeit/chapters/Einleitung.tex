\chapter{Einleitung}

\section{\acl{ARRI}}
Die Bachelorarbeit fand bei der Firma \acf{ARRI} statt. Nach der Gründung 1917 liegt der Hauptsitz von \ac{ARRI} immer noch in München. 
Mittlerweile sind weltweit um die 1500 Mitarbeiter angestellt und die Firma ist einer der führenden Hersteller und Lieferanten in der Film- und Fernsehindustrie.

Die \ac{ARRI} Gruppe ist in fünf Geschäftsbereiche eingeteilt: Kamerasysteme, Licht, Postproduktion, Verleihservice und Operationskamerasystem für die Medizin. \cite{arricorpinfo}

Das Thema dieser Bachelorarbeit wurde im Bereich Kamerasysteme in der Forschungs- und Entwicklungsabteilung erarbeitet.


\section{Hinführung zum Thema}\label{sec:thema}
Damit die Kamera einwandfrei funktioniert müssen Firmware und Software zusammenspielen. Im \ac{geo} werden die verschiedenen Module aus dem \ac{fpga} abgebildet und entsprechend der Kameraeinstellungen die Bildgrößen berechnet. 

In der Software werden dann in verschiedenen Funktionen die einzelnen Register im \ac{fpga} entsprechend gesetzt. Durch die aktuelle Implementierung können keine Module gleichzeitig im \ac{fpga} geupdatet werden. 

%todo plural!
Damit eine modulare Ansteuerung möglich wird, soll eine weitere Abstraktionsebene erstellt werden. Dieses sogenannte \ac{fra} soll die einzelnen Module im Kernel darstellen und entsprechend aus dem Userspace über \ac{ioctl} angesprochen werden können. Zusätzlich werden dann die direkten Hardwarezugriffe aus dem Hauptcode eliminiert. 

Durch die neue Abstraktionsebene soll die Wartbarkeit sowie die Erweiterung der Kamerasoftware in Zukunft ohne tiefere Kenntnisse von \ac{fpga} durch unterschiedliche Entwickler möglich werden. 


%todo kernelversion der kamera und der zitate, erklärung das unterschiedliche nummern, da grundlegend identisch, neuere versionen allerdings mit mehr und ausfühlicheren kommentaren
