% Vorspann der Arbeit
%

\includegraphics[width=45mm]{pictures/Hochschule_Muenchen_Logo.png}
\hfill
\includegraphics[width=45mm]{pictures/ARRI_AG_Corporate_Logo.png}

\vspace*{20mm}
\begin{center}
{\large Hochschule f\"ur angewandte Wissenschaften M\"unchen}\\
{\large Fakult\"at f\"ur Elektrotechnik und Informationstechnik}\\
{\large Bachelorstudiengang Elektrotechnik und Informationstechnik}\\

\vspace*{15mm}
{\huge Bachelorarbeit }    %% oder Masterarbeit
\\

\vspace*{10mm}
{\huge \bfseries{%
                      \par Modulare Ansteuerung eines FPGAs \"uber Software \par
}} 


\vspace*{15mm}
{\Large abgegeben von %
                      Maren Konrad 
%%                    Vorname Name
} 
\\
\end{center}

\vfill
{\large
\begin{tabbing}
\\
Bearbeitungsbeginn: \hspace{3.5cm} \= %
					  12.08.2019
\\
Abgabetermin: \> \= %     
                      12.02.2020
%%                    durch Ende-Datum ersetzen
\\
lfd. Nr. gem\"a{\ss}  Belegschein: \> %
					  1883 
\\
\end{tabbing} }
\thispagestyle{empty}
\cleardoublepage


% Seite 2
\vspace*{20mm}
\begin{center}
{\large Hochschule f\"ur angewandte Wissenschaften M\"unchen}\\
{\large Fakult\"at f\"ur Elektrotechnik und Informationstechnik}\\
{\large Bachelorstudiengang Elektrotechnik und Informationstechnik}\\

\vspace*{15mm}
{\huge Bachelorarbeit }    %% oder Masterarbeit
\\

\vspace*{10mm}
{\huge \bfseries{%
		\par Modulare Ansteuerung eines bildverarbeitenden FPGAs "uber generische Kernelmodule \par
}} % deuscher Langtitel

\vspace*{10mm}
{\huge \bfseries{%
		\par Modular control of an image processing FPGA via generic kernelmodul \par
}} % englischer Langtitel

\vspace*{15mm}
{\Large abgegeben von %
	Maren Konrad 
	%%                    Vorname Name
} 
\\
\end{center}

\vfill
{\large
	\begin{tabbing}
		\\
		Bearbeitungsbeginn: \hspace{3.5cm} \= %     
		12.08.2019
		%%                    durch Start-Datum ersetzen
		\\
		Abgabetermin: \> %     
		12.02.2020
		%%                    durch Ende-Datum ersetzen
		\\
		lfd. Nr. gem\"a{\ss}  Belegschein: \> %
		1883
		%%                    durch Lfd. Nr. ersetzen
		\\
		Betreuer (Hochschule M\"unchen): \> %         
		Prof. Dr. Gerhard Schillhuber
		%%                    hier den Prof. eintragen
		\\
		Betreuer (Extern): \> %         
		Anton Hattendorf
\end{tabbing} }
\thispagestyle{empty}
\cleardoublepage

% Seite 3
\thispagestyle{empty}
Erkl\"arungen des Bearbeiters:\\
\\
Name: Konrad \\
Vorname: Maren \\

1) Ich erkl\"are hiermit, dass ich die vorliegende Bachelorarbeit selbst\"andig verfasst
und noch nicht anderweitig zu Pr\"ufungszwecken vorgelegt habe.

S\"amtliche benutzte Quellen und Hilfsmittel sind angegeben, w\"ortliche und sinngem\"a{\ss}e Zitate sind als solche gekennzeichnet.
\vspace{15mm}

\begin{tabbing}
	M\"unchen, den \today \hspace{10mm} \= \rule{60mm}{0.5pt}\\
	\> {\small Unterschrift}
\end{tabbing}
\vspace{15mm}

\vfill
2) Der Ver\"offentlichung der Bachelorarbeit stimme ich hiermit \textbf{NICHT} zu.
\vspace{15mm}

\begin{tabbing}
	M\"unchen, den \today \hspace{10mm} \= \rule{60mm}{0.5pt}\\
	\> {\small Unterschrift}
\end{tabbing}

\vfill
\cleardoublepage

%Seite 4

\begin{center}
{\Large \bfseries{ Kurzfassung }}\\
\end{center}

%% Hier die Kurzfassung in Deutsch einf�gen
%In dem vorliegenden Bericht geht es um die Verbesserung einer Kamerasoftware. Es wird hierzu eine Abstraktionsebene vorgestellt, die aus mehreren Modulen besteht. Um das Verst\"andnis zu erleichtern wird au\ss{}erdem eine kurze Einf\"uhrung zur Bildkette gegeben und anschlie\ss{}end auf ausgew\"ahlte Module eingegangen. Durch die Abstraktionsebene wird die \"Ubersichtlichkeit und Erweiterung der Software in Zukunft erleichtert werden.

%todo noch ein bissl mehr dazu schreiben
In dem vorliegenden Bericht geht es um die Verbesserung einer Kamerasoftware im Zusammenspiel mit der Hardware. Es wird hierzu eine Abstraktionsebene vorgestellt, die die Zugriffe von der Software auf den FPGA \"ubersichtlicher und einfacher gestalten soll. Der Zugriff auf die Hardware soll nicht mehr direkt \"uber die Adresse stattfinden, sondern abstrahiert werden, sodass die Software lediglich den Namen ben\"otigt um auf das richtige Modul zuzugreifen. Zum Erleichteren des Verst\"andins wird ein kurzer \"Uberblick \"uber die aktuelle Implementierung und eine kurze Einf\"uhrung zur Bildkette gegeben. Au\ss{}erdem wird auf das Betriebssystem Linux und ben\"otigte Komponenten eingegangen. Durch die Abstraktionsebene soll die Wartbarkeit und Erweiterung der Software in Zukunft erleichtert werden.

Dieser Bericht ist f\"ur Leser aus dem Bereich der Elektrotechnik geschrieben.

\vspace*{5mm}
\begin{center}
{\Large \bfseries{ Abstract }}\\
\end{center}

%% Hier die Kurzfassung in Englisch ein
%This report handle the improvement of an camera software. For this, an abstraction level will be introduce, which consists out of a few modules. At first there is a short intorduction to the image processing chain to make the understanding easier and then go into details of selected modules. Because of the abstraction level the clarity and extensions of the software will be easier in future. 

This report targets readers having an electric engineering background.

\pagenumbering{Roman} 	
\renewcommand{\thechapter}{\Roman{chapter}} 

\cleardoublepage
%% Inhaltsverzeichnis
\tableofcontents
\clearpage



